\documentclass{article}
\usepackage[utf8]{inputenc}

\title{Opensees Resumen}
\author{Francisco Nazar}
\date{ }

\usepackage{natbib}
\usepackage{graphicx}

\begin{document}

\maketitle
\section{Tcl}

\begin{itemize}
\item El comando \bf incr \rm incrementa el valor de una variable. Ejemplo:
\begin{verbatim}
set a 1
incr a
\end{verbatim}

\item La sintaxis básica para un comando de Tcl es 
\begin{verbatim}
command $arg1 $arg2
\end{verbatim}


\item Tcl permite comandos anidados: 
\begin{verbatim}
command [nested-command1] [nested-command2]
\end{verbatim}


\item El comando más básico de Tcl es \tt set \rm 
\begin{verbatim}
set variable $value
\end{verbatim}
por ejemplo 

\begin{verbatim}
set a 5
\end{verbatim}

\item Para comentar se usa el símbolo #

\item Para comentar se usa el símbolo #
\begin{verbatim}
set variable $value
\end{verbatim}



\end{itemize}


\end{document}
